\documentclass[12pt,english]{simplecv}
\usepackage[T1]{fontenc}
\usepackage[latin9]{inputenc}
\usepackage{geometry}
\geometry{verbose,tmargin=1in,bmargin=1in,lmargin=1in,rmargin=1in}
\setcounter{secnumdepth}{0}
\setcounter{tocdepth}{0}
\usepackage{color}
\usepackage[unicode=true, 
pdfusetitle,
bookmarks=true,
bookmarksnumbered=false,
bookmarksopen=false,
breaklinks=true,
pdfborder={0 0 1},
backref=false,
colorlinks=true]
{hyperref}
\hypersetup{urlcolor=blue, linkcolor=blue, citecolor=blue}

\makeatletter
%%%%%%%%%%%%%%%%%%%%%%%%%%%%%% User specified LaTeX commands.
%% You can modify the fonts used in the document be using the
%% following macros. They take one parameter which is the font
%% changing command.
%% \headerfont: the font used in both headers.
%%              Defaults to sans serif.
%% \titlefont:  the font used for the title.
%%              Defaults to \LARGE sans-serif semi bold condensed.
%% \sectionfont: the font used by \section when beginning a new topic.
%%              Defaults to sans-serif semi bold condensed.
%% \itemfont:   the font used in descriptions of items.
%%              Defaults to sans-serif slanted.
% to make your name even bigger, uncomment the following line:
% \titlefont{\Huge}
%%
%% You can modify the following parameters using \renewcommand:
%% \topicmargin: the left margin inside topics.
%%               Defaults to 20% of the column width (0.20\columnwidth).
% To get more room for left column of Topic layouts, uncomment following line:
% \renewcommand{\topicmargin}{0.3\columnwidth}

\makeatother

\usepackage{babel}
\begin{document}

\leftheader{Purok 4, Lunga, Valencia\\Negros Oriental 6215\\Philippines}

\rightheader{aubreymoore2013@gmail.com}

\title{Aubrey Moore}
\maketitle

\section{Education}
\begin{topic}
\item [{1988}]Ph.D., Entomology, University of Hawaii, Honolulu, Hawaii
\item [{1984}] M.S., Entomology, Michigan State University, East Lansing, Michigan
\item [{1979}] B.Sc., Integrated Science Studies, Carleton University, Ottawa, Ontario 
\end{topic}

\section{Professional Experience}
\begin{topic}
\item [{2008-2023}] Extension Entomologist, Cooperative Extension Service, University of Guam, Guam 
\item [{2003-2008}] Research Associate, College of Natural \& Applied Sciences, University of Guam, Guam 
\item [{1999-2003}] Pesticide Evaluator, Pest Management Regulatory Agency, Health Canada, Ottawa, ON
\item [{1998-1999}] Entomologist, School of Agriculture \& Life Sciences,
Northern Marianas College, Saipan 
\item [{1992-1997}] Research Director, School of Agriculture \& Life Sciences,
Northern Marianas College, Saipan 
\item [{1991-1992}] Entomologist, Northern Mariana Islands Department of
Natural Resources, Saipan
\item [{1990-1991}] Entomologist, USDA Agricultural Development in the
American Pacific Project, Guam \& Maui
\item [{1989-1990}] Research Associate, University of Hawaii Agricultural
Experiment Station, Maui, Hawaii
\item [{1988}] Post-doctoral Fellow, Hawaiian Evolutionary Biology Program,
University of Hawaii, Honolulu, Hawaii 
\item [{1985\textendash 1988}] Graduate Assistant, Department of Entomology,
University of Hawaii, Honolulu, Hawaii 
\item [{1985-1986}] Programmer/consultant, University of Hawaii Computing
Centre, Honolulu, Hawaii
\item [{1984}] Research Associate, Department of Entomology, Michigan State
University, East Lansing, MI
\item [{1984}] Entomologist, Insect and Rodent Control Section, Michigan
Dept. of Public Health, Lansing, MI
\item [{1981-1984}] Graduate Assistant, Department of Entomology, Michigan
State University, East Lansing, MI 
\item [{1979-1981}] Research Tech., Forest Pest Management Institute, Environment
Canada, Sault Ste. Marie, ON
\item [{1975-1979}] Research Technician, Chemical Control Research Institute,
Environment Canada, Ottawa, ON
\end{topic}

%\section{Professional Memberships}
%
%\noindent Entomological Society of America\\
%Hawaiian Entomological Society\\
%Florida Entomological Society\\
%Pacific Science Association\\
%Sigma Xi Research Fraternity

\section{Publications}

\subsection{Book Chapters}
\begin{thebibliography}{10}

\bibitem{} Cave, R. D., Moore, A., \& Wright, M. G. 2022. Biological Control of the Cycad Aulacaspis Scale, \textit{Aulacaspis yasumatsui}. In Contributions of Classical Biological Control to U.S. Food Security, Forestry, and Biodiversity. \url{https://github.com/aubreymoore/CAS/raw/main/CAS_Biocontrol.pdf}

\bibitem{key-6}Moore, A. \& J. A. Tenorio 2006. Our Islands' Insects
and Their Relatives. In \emph{Island Ecology and Resource Management}.
Editor: J. Furey; Publisher: Northern Marianas College Press.

\bibitem{key-30}Schreiner, I., L. Yudin, A. Moore \& D. Nafus 1998.
Management of Insects and Mites. In \emph{Guam Cucurbit Guide}. Editors:
L. Yudin \& R. Schlub; Publisher: College of Agriculture \& Life Sciences,
University of Guam.

\bibitem{key-29}Hunter, W. B., D. E. Ullman \& A. Moore 1994. Electronic
Monitoring: Characterizing the Feeding Behavior of Western Flower
Thrips (Thysanoptera: Thripidae). in \emph{History, Development, and
Application of AC Electronic Insect Feeding Monitors}. Editors: M.
M. Ellsbury, E. A. Backus \& D. L. Ullman; Publisher: Entomological
Society of America.
 
\end{thebibliography}

\subsection{Journal Articles}

\begin{thebibliography}{10}
%2023
\bibitem{} Caasi, J. A. S., Guerrero, A. L., Yoon, K., Aquino, L. J. C., Moore, A., Oh, H., Rychtar, J., \& Taylor, D. (2023). A mathematical model of invasion and control of coconut rhinoceros beetle \textit{Oryctes rhinoceros} (L.) in Guam. Journal of Theoretical Biology, 111525. \url{https://doi.org/10.1016/j.jtbi.2023.111525}

\bibitem{} Paudel, S., Jackson, T. A., Mansfield, S., Ero, M., Moore, A., \& Marshall, S. D. G. (2023). Use of pheromones for monitoring and control strategies of coconut rhinoceros beetle (\textit{Oryctes rhinoceros}): A review. Crop Protection, 174, 106400. \url{https://doi.org/10.1016/j.cropro.2023.106400}

%2022
\bibitem{} Moore, A., \& Siderhurst, M. (2022). Proposal for detecting coconut rhinoceros beetle breeding sites using harmonic radar. Research Ideas and Outcomes, 8, e86422. \url{https://doi.org/10.3897/rio.8.e86422}

%2021
\bibitem{} Siderhurst, M. S., Moore, A., Quitugua, R., \& Chang, E. B. (2021). Effects of Ultraviolet Light and Pheromone Release Rate in Trapping Coconut Rhinoceros Beetles, \textit{Oryctes rhinoceros} (Coleoptera: Scarabaeidae), on Guam. \url{http://scholarspace.manoa.hawaii.edu/handle/10125/81413}

\bibitem{} Moore, A. (2021). Research Idea: Using Mosquitoes to Detect Brown Treesnakes. \url{https://doi.org/10.5281/zenodo.7637639}
%2019	
%2018
\bibitem{} Moore, A. (2018). The Guam Coconut Rhinoceros Beetle Problem: Past, Present and Future. Zenodo. \url{https://doi.org/10.5281/zenodo.1185371}

\bibitem{} Manuel, J., Tennent, W. J., Buden, D. W., \& Moore, A. (2018). First record of \textit{Doleschallia tongana} (Lepidoptera: Nymphalidae) for Guam Island. F1000Research, 7, 366. \url{https://doi.org/10.12688/f1000research.14316.1}

\bibitem{} Moore, A., Barahona, D. C., Lehman, K. A., Skabeikis, D. A., Iriarte, I. R., Jang, E. B., \& Siderhurst, M. S. (2017). Judas beetles: Discovering cryptic breeding sites by radio-tracking coconut rhinoceros beetles, \textit{Oryctes rhinoceros} (Coleoptera: Scarabaeidae). Journal of Environmental Entomology, 46(1), 92-99. \url{https://doi.org/10.1093/ee/nvw152}

\bibitem{} Marshall, S. D. G., Moore, A., Vaqalo, M., Noble, A., \& Jackson, T. A. (2017). A new haplotype of the coconut rhinoceros beetle, \textit{Oryctes rhinoceros}, has escaped biological control by \textit{Oryctes rhinoceros} nudivirus and is invading Pacific Islands. Journal of Invertebrate Pathology, 149, 127-134. \url{https://doi.org/10.1016/j.jip.2017.07.006}

\bibitem{} Moore, A., Quitugua, R., Iriarte, I., Melzer, M., Watanabe, S., Cheng, Z., \& Barnes, J. M. (2016). Movement of Packaged Soil Products as a Dispersal Pathway for Coconut Rhinoceros Beetle, \textit{Oryctes rhinoceros} (Coleoptera: Scarabaeidae) and Other Invasive Species. Proceedings of the Hawaiian Entomological Society, 48, 21-22. Retrieved from \url{http://scholarspace.manoa.hawaii.edu/handle/10125/42743}

\bibitem{} Moore, A., Jackson, T., Quitugua, R., Bassler, P., \& Campbell, R. (2015). Coconut rhinoceros beetles ( Coleoptera: Scarabaeidae ) develop in arboreal breeding sites in Guam. Florida Entomologist, 98(3), 1012-1014. Retrieved from \url{http://journals.fcla.edu/flaent/article/download/84794/84044}

\bibitem{} Moore, A., Watson, G., \& Bamba, J. (2014). First record of eggplant mealybug, \textit{Coccidohystrix insolita} (Hemiptera: Pseudococcidae), on Guam: Potentially a major pest. Biodiversity Data Journal, 2. \url{https://doi.org/10.3897/BDJ.1.e1042}

\bibitem{key-1}Marler, T.E, A. Moore, and R. Miller 2013. Vertical
stratification in predation of armored scale on \emph{Cycas micronesica}
seedlings. HortScience 48(1) 60-62.

\bibitem{key-16}Marler, TE, Wiecko G, Moore A. 2012. Application
of game theory to the interface between militarization and environmental
stewardship in the Mariana Islands. Commun Integr Biol. 5:193-195
.URL: \url{http://dx.doi.org/10.4161/cib.18889}

\bibitem{key-1}Marler, T. E., L. S. Yudin and A. Moore 2011. \emph{Schedorhinotermes
longirostris} (Isoptera: Rhinotermitidae) invades Guam: yet another
assault on the endemic \emph{Cycas micronesica}. Florida Entomologist
94: 699-700.

\bibitem{key-2}Marler, T. E. and A. Moore 2011. Military threats
to terrestrial resources not restricted to wartime: a case study from
Guam. Journal of Environmental Science and Engineering (USA) 5: 1198-1214.

\bibitem{key-3}Marler, T. E. and A. Moore 2010. Cryptic scale infestations
on \emph{Cycas revoluta} facilitate scale invasions. HortScience 45(5):
837-839.

\bibitem{key-4}Van Driesche, R.G., Carruthers, R.I., Center, T.,
Hoddle, M.S., Hough-Goldstein, J., Morin, L., Smith, L., Wagner, D.L.,
Blossey, B., Brancatini, V., Casagrande, R., Causton, C.E., Coetzee,
J. A., Cuda, J., Ding, J., Fowler, S.V., Frank, J.H., Fuester, R.,
Goolsby, J., Grodowitz, M., Heard, T.A., Hill, M.P., Homann, J.H.,
Huber, J., Julien, M., Kairo, M.T.K., Kenis, M., Mason, P., Medal,
J., Messing, R., Miller, R., Moore, A., Neuenschwander, P., Newman,
R., Norambuena, H., Palmer, W.A., Pemberton, R., Perez Panduro, A.,
Pratt, P.D., Rayamajhi, M., Salom, S., Sands, D., Schooler, S., Sheppard,
A., Shaw, R., Schwarzl nder, M., Tipping, P.W., van Klinken, R.D.,
2010. Classical biological control for the protection of natural ecosystems:
past achievements and current efforts. Biological Control. Biological
Control 54 Supplement 1: S2-S33.

\bibitem{key-6}Mankin, R.W., A. Moore 2010. Acoustic detection of
\emph{Oryctes rhinoceros} (Coleoptera: Scarabaeidae: Dynastinae) and
\emph{Nasutitermes luzonicus} (Isoptera: Termitidae) in palm trees
in urban Guam. Journal of Economic Entomology. 103: 1135-1143.

\bibitem{key-27}Moore, A. \& L. R. Barber 2008. Wiki based fact sheets.
Journal of Extension 46(3).

\bibitem{key-26}Moore, A. \& R. H. Miller 2008. \emph{Daphnis nerii}
(Lepidoptera: Sphingidae), a new pest of oleander on Guam. Proc. Hawaiian
Entomol. Soc. 40: 67-70.

\bibitem{key-24}Zack, R.S., A. Moore \& R.H. Miller 2007. First record
of a pygmy backswimmer (Hemiptera: Pleidae) from Micronesia. Zootaxa
1617: 67-68.

\bibitem{key-23}Williams, D. J., P. J. Gullan, K. Englberger \& A.
Moore 2006. Report on the scale insect, \emph{Icerya imperatae}, Rao
(Hemiptera: Coccoidea: Margarodidae) seriously infesting grasses in
the Republic of Palau. Micronesica 38(2): 269-274.

\bibitem{key-22}Moore, A. \& R. Miller. 2002. Automated identification
of optically sensed aphid wingbeat waveforms. Ann. Entomol. Soc. Am.
95(1): 1-8.

\bibitem{key-20}Caprio, M.A., J.-X. Huang, M.K. Faver \& A. Moore.
2001. Characterization of male and female wingbeat frequencies in
the \emph{Anopheles quadrimaculatus} complex in Mississippi. Journal
of the American Mosquito Control Association: 17(3): 186-189.

\bibitem{key-19}Moore, A. 1998. Development of a data acquisition
system for long-term outdoor recording of insect flight activity using
a photosensor. Proceedings of the 13th Conference on Biometeorology
and Aerobiology, Albuquerque, New Mexico.

\bibitem{key-18}Chiu, C. H. \& A. Moore. 1993. Biological control
of the Philippine lady beetle, \emph{Epilachna philippinensis} Dieke
(Coleoptera: Coccinelidae), on solanaceous plants by introducing the
parasitoid, Pediobius foveolatus Crawford (Hymenoptera: Eulophidae),
on Saipan. Micronesica, Supplement No. 4: 79-80. 

\bibitem{key-17}Moore, A., B. E. Tabashnik \& M. D. Rethwisch. 1992.
Sublethal effects of fenvalerate on adults of the diamondback moth.
J. Econ. Entomol. 85: 1624-1627. 

\bibitem{key-16}Moore, A. 1991. Automated identification of insects
in flight. Micronesica. Supplement No. 3: 129-133. 

\bibitem{key-15}Moore, A. 1991. Artificial neural network trained
to identify mosquitoes in flight. J. Insect Behavior. 4: 391-395. 

\bibitem{key-14}Moore, A., B. E. Tabashnik, \& J. D. Stark 1989.
Leg autotomy: a novel mechanism of protection against insecticide
poisoning in the diamondback moth (Lepidoptera: Plutellidae). J. Econ.
Entomol. 82: 1295-1298. 

\bibitem{key-13}Moore, A. and B. E. Tabashnik 1989. Leg autotomy
of adult diamondback moth (Lepidoptera: Plutellidae) in response to
tarsal contact with insecticide residues. J. Econ. Entomol. 82: 381-384. 

\bibitem{key-12}Moore, A. 1988. Auto-amputation in diamondback moths:
a new form of insecticide resistance? Pacific Science 42: 128-129. 

\bibitem{key-11}Moore, A., J. R. Miller, B. E. Tabashnik and S. H.
Gage 1986. Automated identification of flying insects by analysis
of wingbeat frequencies. J. Econ. Entomol. 79: 1703-1706. 

\bibitem{key-10}O. N. Morris and A. Moore 1983. Relative potencies
of \emph{Bacillus thuringiensis} for larvae of the spruce budworm,
Choristoneura fumiferana (Lepidoptera: Tortricidae). Can. Entomol.
115: 815-822. 

\bibitem{key-9}O. N. Morris and A. Moore 1983. Changes in spruce
budworm, \emph{Choristoneura fumiferana} (Lepidoptera: Tortricidae),
biomass in stands treated with commercial Bacillus thuringiensis var.
kurstaki. Can. Entomol. 115:4. 

\bibitem{key-8}Moore, A. and O. N. Morris 1982. An improved technique
for dosing larvae of the spruce budworm, \emph{Choristoneura fumiferana}
(Lepidoptera: Tortricidae) with measured amounts of \emph{Bacillus
thuringiensis} var. \emph{kurstaki}. Can. Entomol. 114:89-91. 

\end{thebibliography}

\section{Presentations}

\begin{thebibliography}{999}

%2023

\bibitem{} Moore, A. (2023, April 15). Interesting facts about chili pepper. University of Guam. \url{https://aubreymoore.github.io/pika/}

\bibitem{} Sugimoto, K., Yamauchi, M., Moore, A., Marshall, S. D. G., Kojima, A., \& Nakai, M. (2023, September). Establishment of a bioassay method for the Formosan beetle to compare \textit{Oryctes rhinoceros} nudivirus susceptibility in different regional populations. Entomological Society of Japan, Saga University, Saga City, Japan.

%2022
\bibitem{} Cave, R. D., \& Moore, A. (2022, March 8). Biological control of cycad aulacaspis scale (webinar). \url{https://aubreymoore.github.io/CAS-biocontrol-seminar/}

\bibitem{} Moore, A. (2022, April). The Invasive Species Problem on Guam. Western Plant Diagnostics Network Annual Meeting, Davis, California. https://aubreymoore.github.io/WPDN2022/

\bibitem{} Grasela, J. J., \& Moore, A. (2022, June 21). Preliminary detection of \textit{Wolbachia} in the coconut rhinoceros beetle \textit{Oryctes rhinoceros} (Coleoptera: Scarabaeidae) from Guam, Palau, and Taiwan. \url{https://doi.org/10.5281/zenodo.6672841}

\bibitem{} Grasela, J. J., \& Moore, A. (2022, June 21). Preliminary efforts to establish a continuous cell line for coconut rhinoceros beetle (\textit{Oryctes rhinoceros}, Coleoptera: Scarabaeidae). \url{https://doi.org/10.5281/zenodo.6672831}

\bibitem{} Moore, A., Quitugua, R., \& Dulla, G. (2022, October 6). Overview of Invasive Species Issues on Guam, Pacific Ecological Security Conference, Palau, October 6, 2022. \url{https://doi.org/10.5281/zenodo.7593110}

%2021

\bibitem{} Moore, A. (2020, February 11). Predicting invasive species arrivals on Guam. Forestry Workshop on Invasive Insects, University of Guam, Mangilao, Guam. https://aubreymoore.github.io/guam-ias-bolo

\bibitem{} Moore, A. (2021). How Bad is Guam?s Invasive Species Problem?: A Global Perspective. Marianas Terrestrial Conservation Conference, Guam. \url{https://aubreymoore.github.io/top-10-most-costly-ias-mtcc/}

\bibitem{} Moore, A. (2021, July 30). Biological Invasion of Guam?s Forests. Guam Soil and Water Conservation Districts 2021 Educator?s Symposium: Healthy Forests, Healthy Communities, Guam. \url{https://aubreymoore.github.io/albi345-slides/SWCD-2021-07-30/}

\bibitem{} Moore, A. (2021, December). Presentation: Using harmonic radar to track the greater banded hornets to their nests so that they can be destroyed. Guam Beekeepers Association Meeting, Jeff;s Pirates Cove, Ipan, Guam.

\bibitem{} Barrera, G., Marshall, S., Moore, A., \& Jackson, T. (2021, July 21). Electron microscopy study confirms infection of coconut rhinoceros beetle (CRB-G) gut cells by OrNV -V23B. (Poster) Abstracts?2021 International Congress on Invertebrate Pathology and Microbial Control \& 53rd Annual Meeting of the Society for Invertebrate Pathology. Le Studium Conference (Virtual), Tours France. P 137. \url{https://www.researchgate.net/publication/353356673_Electron\_microscopy\_Congress\_on\_Invertebrate\_Pathology\_and\_Microbial\_Control\_5}

\bibitem{} Marshall, S. D. G., Barrera, G., Villamizar, L. F., Suda, G., Moore, A., Grasela, J. J., Scotti, P. D., \& Jackson, T. A. (2021, June 21). Production of \textit{Oryctes} nudivirus (OrNV) through the DSIR-Ha-1179 \textit{Heteronychus arator} cell line. (Poster) Abstracts?2021 International Congress on Invertebrate Pathology and Microbial Control \& 53rd Annual Meeting of the Society for Invertebrate Pathology. Le Studium Conference (Virtual), Tours France.

\bibitem{} Moore, A. (2021, February 23). CRB Biology: Know Your Enemy. CNMI CRB Project Teleconference. \url{https://github.com/aubreymoore/CRB-CNMI/raw/main/CRB-Biology.pdf}

%2020
\bibitem{} Moore, A., \& Jackson, T. (2020, December 9). Automated roadside video surveys for detecting and monitoring coconut rhinoceros beetle damage to coconut palms. Presented at the Annual Meeting of the CRB-G Action Group. Annual meeting of the CRB-G Action Group. \url{https://aubreymoore.github.io/crb-roadside-slides}

%2019
\bibitem{} Moore, A. (2019, March). 2019 Forest Service Review of University of Guam Projects. \url{https://github.com/aubreymoore/2019-Forest-Service-Review/raw/master/2019%20Forest%20Service%20Review.pdf}

\bibitem{} Moore, A. (2019, November 11). Status of a Major Outbreak of Coconut Rhinoceros Beetle,. Oryctes rhinoceros biotype G, on Guam and Attempts at Establishing Biological Control. XIX International Plant Protection Congress, Hyderabad, India. \url{https://github.com/aubreymoore/IAPPS-2019-Presentation/raw/master/Moore_IAPPS-2019.odp}

\bibitem{} Marshall, S. D. G. (2019, November 11). The challenge of coconut rhinoceros beetle (Oryctes rhinoceros) to palm production and prospects for control in a changing world. XIX International Plant Protection Congress, Hyderabad, India.

\bibitem{} Moore, A. (2019, March). Entomology section: 17th annual quarantine training workshop, Guam 2019. \url{https://osf.io/ndz2h}

%2018




\bibitem{}Moore A. 2018. Failed Attempts to Establish IPM for Asian Cycad Scale and Coconut Rhinoceros Beetle on Guam. Annual Meeting of the Entomological Society of America; 2018 Nov; Vancouver, BC, Canada.

\bibitem{} Moore, A., Marshall, S. D. G., Quitugua, R., \& Iriarte, I. R. (2018, September 13). Attempted microbial control of coconut rhinoceros beetle, \textit{Oryctes rhinoceros}, biotype G on Guam using \textit{Oryctes rhinoceros} nudivirus and \textit{Metarhizium majus}. 51st Annual Meeting of the Society for Invertebrate Pathology and International Congress on Invertebrate Pathology and Microbial Control, Gold Coast, Australia. \url{https://github.com/aubreymoore/SIP2018}
 
\bibitem{}Rosario C, Miller R, Moore A, Sablan, L. 2018. Greater banded hornet (\textit{Vespa tropica}) established at several locations on Guam. Annual Meeting of the Entomological Society of America; 2018 Nov; Vancouver, BC, Canada. 
	
\bibitem{}Moore, A. 2018. Coconut Rhinoceros Beetle Update. Regional Invasive Species Council, Guam, 2018 Sep 20. 

\bibitem{}Moore, A. 2018. Guam Biodiversity Inventory. Regional Invasive Species Council Meeting at the Plant Inspection Facility, Tiyan, Guam 2018 Sep 21. 

\bibitem{}Moore, A. 2018. The Coconut Rhinoceros Beetle Outbreak on Guam: What Can Be Done About It? [Internet]. 2018 Sep 22. Available from: \url{https://ndownloader.figshare.com/files/13141172}

\bibitem{}Blas AL, Quitugua R, Moore A 2018. Protecting a cultural icon and food resource: Current research and status of Coconut palm in Guam and the Northern Marianas [Internet]. Joint Meeting of the American Phytopathological Society (APS), Pacific Division and Conference on Soilborne Plant Pathogens (CSPP); 2018 Jun 27 [cited 2018 Aug 25]; Portland, Oregon. Available from: \url{https://www.apsnet.org/members/divisions/pac/meetings/Documents/APS\_PacificDivisionCSPP\_2018\_PROGRAM%20SCHEDULE.pdf}
	
\bibitem{}Deloso BE, Moore A, Marler TE 2018. Parasitoid Surveys in Cycad Habitats on Guam. American Society for Horticulture Science 2018 Annual Conference; 2018 Aug 3 [cited 2018 Aug 25]; Washington, D.C. Available from: \url{https://ashs.confex.com/ashs/2018/meetingapp.cgi/Paper/28523}
	
\bibitem{}Marshall SDG, Moore A, Ero M, Fanai C, Vaqalo M, Jackson TA 2018. Progress with control of a virus resistant coconut rhinoceros beetle. 51st Annual Meeting of the Society for Invertebrate Pathology and International Congress on Invertebrate Pathology and Microbial Control; 2018 Sep 13; Gold Coast, Australia. 
	
\bibitem{}Moore A 2018. Biological Invasion of Guam. WEDA/WAAESD Joint Summer Meeting; 2018 Jul 11 [cited 2018 Jul 20]; Guam. Available from: \url{https://github.com/aubreymoore/Guam-Bioinvasion-July-2018/raw/master/compress\_biological\_invasion\_of\_guam\_July\_2018.pdf}
	
\bibitem{}Moore A. 2018. Biological Invasion of Guam. 2018 Coconut Rhinoceros Beetle Training for CNMI; 2018 Jul 30; UOG, Guam. 
	
\bibitem{}Moore A. 2018. Building a Terrestrial Biodiversity Inventory for Guam. Guam Island Sustainability Conference; 2018 Apr 26 [cited 2018 May 30]; Tumon Bay, Guam. Available from: \url{https://figshare.com/articles/Building\_a\_Terrestrial\_Biodiversity\_Inventory\_for\_Guam/6188315}
	
\bibitem{}Moore A. 2018. Building a Terrestrial Biodiversity Inventory for Guam [Internet]. Oral presentation presented at: Second Annual Digital Data in Biodiversity Research Conference; 2018 [cited 2018 May 30]; Berkeley, CA. Available from: \url{https://figshare.com/articles/Building\_a\_Terrestrial\_Biodiversity\_Inventory\_for\_Guam/6188315}
	
\bibitem{}Moore A. 2018. Coconut Rhinoceros Beetle Invasion of Guam. 2018 Coconut Rhinoceros Beetle Training for CNMI; 2018 Jul 30; UOG, Guam. 
	
\bibitem{}Moore A. 2018. Free Cell Phone Apps for Pest Surveys. 2018 Coconut Rhinoceros Beetle Training for CNMI; 2018 Aug 9; UOG, Guam. 
	
\bibitem{}Moore A, Marshall SDG, Quitugua R, Iriarte IR 2018. Attempted microbial control of coconut rhinoceros beetle, \textit{Oryctes rhinoceros}, biotype G on Guam using \textit{Oryctes rhinoceros} nudivirus and \textit{Metarhizium majus}. 51st Annual Meeting of the Society for Invertebrate Pathology and International Congress on Invertebrate Pathology and Microbial Control; 2018 Sep 13; Gold Coast, Australia. Available from: \url{https://www.zotero.org/aubreymoore/items/7VDF7QFR/file}

% 2017

\bibitem{}Moore A. 2017. Access to Information on Forest Insect Pests in Micronesia. 2017 Pacific Island Forestry Professionals Workshop; 2017 Apr 4; Tumon Bay, Guam. 
 
\bibitem{}Moore A. 2017. Biological Control of Cycad Scale, Aulacaspis yasumatsui, Attacking Guam?s Endemic Cycad, Cycas micronesica [Internet]. 2017 Pacific Island Forestry Professionals Workshop; 2017 Apr 4 [cited 2017 Apr 3]; Tumon Bay, Guam. Available from: \url{https://github.com/aubreymoore/Guam-Forestry-Workshop-Resources/raw/master/CycadScaleBiocontrolChile.pdf}
 
\bibitem{}Moore A. 2017. Biological Invasion of Forests on Guam and Other Islands in Micronesia [Internet]. 2017 Pacific Island Forestry Professionals Workshop; 2017 Apr 4 [cited 2017 Apr 3]; Tumon Bay, Guam. Available from: \url{https://aubreymoore.github.io/PDF\_to\_Reveal/reveal.js/slides.html}
 
\bibitem{}Moore A. 2017. Biological Invasion of Guam [Internet]. 2017 Pacific Island Forestry Professionals Workshop; 2017 Apr 4 [cited 2017 Apr 3]; Tumon Bay, Guam. Available from: \url{https://aubreymoore.github.io/PDF\_to\_Reveal/reveal.js/slides.html}
 
\bibitem{}Moore A. 2017. Coconut Rhinoceros Beetle [Internet]. Extension and Outreach Monthly Meeting; 2017 Apr 7; University of Guam. Available from: \url{https://aubreymoore.github.io/extalk-APR2017/EXTALK\_APR2017.html}
 
\bibitem{}Moore A. 2017.  Impact of climate change on coconut rhinoceros beetle outbreaks in the Pacific [Internet]. Guam Extension and Outreach Climate Forum; 2017 Oct 26; Guam. Available from: \url{https://github.com/aubreymoore/crb-climate-change/blob/master/crb-climate-connection.pdf}
 
\bibitem{}Moore A. 2017. Invasion of Guam by the Coconut Rhinoceros Beetle, \emph{Oryctes rhinoceros} (Linnaeus 1758). 2017 Island Sustainability Conference; 2017; Guam. 
 
\bibitem{}Moore A 2017. The coconut rhinoceros beetle invasion of Guam: An unprecedented disaster [Internet]. 2017 Pacific Island Forestry Professionals Workshop; 2017 Apr 4 [cited 2017 Apr 3]; Tumon Bay, Guam. Available from: The coconut rhinoceros beetle invasion of Guam: An unprecedented disaster
 
\bibitem{}Moore A. 2017. Using free Cell Phone Apps for Forest Pest Surveys. 2017 Pacific Island Forestry Professionals Workshop; 2017 Apr 4; Tumon Bay, Guam. 

% 2016
 
\bibitem{}Aubrey Moore 2016. Discovery of the Coconut Rhinoceros Beetle Guam Biotype and Implications for Global Control [Internet]. Future proofing the palm industries: Limiting damage by existing (CRB-P) and invasive (CRB-G) coconut rhinoceros beetle (Oryctes rhinoceros) in the Pacific; 2016 Jun; Suva, Fiji. Available from: \url{http://guaminsects.net/GISC\_NOV2015/GISC\_NOV2015/Moore\_ESA\_PB\_APR2016.html}

\bibitem{}Aubrey Moore 2016. Guam Report. National Plant Diagnostics Network Meeting; 2016 Mar; Washington, D.C.

\bibitem{}Marshall SDG, Vaqalo M, Moore A, Quitugua R, Jackson TA 2016. Detection of an invasive biotype of Oryctes rhinoceros (L.) in the Pacific [Internet]. XXV International Congress of Entomology; 2016 Sep 26; Orlando, FL. Available from: \url{https://aubreymoore.github.io/CRB-G-ICE2016/Paper95540.html}

\bibitem{}Moore A 2016. Biological Invasion of Guam. Micronesia Plant Pest Quarantine Workshop; 2016 Mar; Guam. 

\bibitem{}Moore A 2016. Discovery of the Coconut Rhinoceros Beetle Guam Biotype and Implications for Global Control [Internet]. Entomological Society of America Pacific Branch Annual Meeting; 2016 Apr 5 [cited 2016 Apr 17]; Honolulu, Hawaii. Available from: \url{http://guaminsects.net/GISC\_NOV2015/GISC\_NOV2015/Moore\_ESA\_PB\_APR2016.html}

\bibitem{}Moore A. 2016. Update on the Guam Coconut Rhinoceros Beetle Infestation. Micronesia Plant Pest Quarantine Workshop; 2016 Mar; Guam. 

\bibitem{}Moore A. 2016. Update on the Guam Coconut Rhinoceros Beetle Infestation. National Plant Diagnostics Network Conference; 2016 Mar; Washington, D.C. 

\bibitem{}Moore A, Quitugua R, Jackson TA, Marshall SDG, Siderhurst MS 2016. The rhinoceros beetle invasion of Guam: An unprecedented disaster [Internet]. XXV International Congress of Entomology; 2016 Sep 26; Orlando, FL. Available from: \url{https://aubreymoore.github.io/CRB-G-ICE2016/Paper94967.html}

% 2015
\bibitem{}Ares MA, Meneses N, Smith A, Moore A, Benford R. Molecular Identification of a Lepidopteran Herbivore on a Critically Endangered Tree. In Northern Arizona Undergraduate Symposium 2015; 2015. Available from: \url{http://guaminsects.net/anr/sites/default/files/Serianthes Herbivore Ares 2015 final(1).pdf}

\bibitem{}Marshall SDG, Vaqalo M, Moore A, Quitugua R, Jackson TA. A new invasive biotype of the coconut rhinoceros beetle (\emph{Oryctes rhinoceros}) has escaped from biocontrol by \emph{Oryctes rhinoceros} nudivirus. In: International Congress on Invertebrate Pathology and Microbial Control and the 48th Annual Meeting of the Society for Invertebrate Pathology, 2015. Available from: \url{http://www.sipmeeting.org/van1/SIP2015-Full Program.pdf}

\bibitem{}Moore A, Quitugua R. Coconut Rhinoceros Beetle Trap Improvements. In: Pacific Entomology Conference, Honolulu. 2015 Apr 1. Available from: \url{http://guaminsects.net/anr/sites/default/files/pec2015-improved-traps.pdf}

\bibitem{}Moore A. 2015. Biosecurity for Guam in the New Millenium: Are We More Secure?. Pacific Entomology Conference, Honolulu, 2015 Apr 1. Available from: \url{https://zenodo.org/record/165694}

\bibitem{}Moore A. 2015. Failure Analysis of the Guam Coconut Rhinoceros Beetle Eradication Project. Pacific Entomology Conference, Honolulu. Available from: \url{https://zenodo.org/record/165762}

\bibitem{}Moore A. 2015. Update on the Guam Coconut Rhinoceros Beetle Infestation. Pacific Plant Protection Organization; 2015 Sep; Nadi, Fiji. 

\bibitem{}Moore A. 2015. Update on the Guam Coconut Rhinoceros Beetle Situation for the Guam Invasive Species Council. Guam Invasive Species Council Meeting, November 15, 2015. Available from: \url{http://guaminsects.net/GISC\_NOV2015/GISC\_NOV2015/}

% 2014

\bibitem{}Moore A. Biological invasion of forests on Guam and other islands of Micronesia. In: 65th Western Forest Insect Work Conference. Sacramento, California; 2014. 

\bibitem{}Moore A. Evaluation of a Scratchpad template as an online database for the University of Guam insect collection. In: iDigBio Biodiversity Collections Digitization in the Pacific Workshop [Internet]. Honolulu, Hawaii; 2014. Available from: \url{https://www.idigbio.org/wiki/images/a/aa/Scratchpads_iDigBio-part1.pdf}

\bibitem{}Moore A, Quitugua R, Siderhurst M, Jang E. Improved traps for the coconut rhinoceros beetle, \emph{Oryctes rhinoceros}. In: Entomological Society of America [Internet]. Portland, OR; 2014. Available from: \url{http://guaminsects.net/anr/sites/default/files/Moore_1957_2.pdf}
	
\bibitem{}Marshall S, Moore A, Campbell R, Quitugua R, Jackson T. 2014. \emph{Oryctes rhinoceros} population diversity and potential implications for control using \emph{Oryctes} nudivirus. 47th Annual Meeting of the Society for Invertebrate Pathology and International Congress on Invertebrate Pathology and Microbial Control; 2014 Aug; Mainz, Germany. Available from: \url{http://www.sipweb.org/docs/Program%20and%20Abstracts%202014.pdf}

\bibitem{}Moore A. Insects Attacking \emph{Serianthes nelsonii} 2014. Available from: \url{https://github.com/aubreymoore/presentations/raw/master/SerianthesInsectPests/SerianthesInsectPest.pdf}

\bibitem{}Moore A, Quitugua R, Siderhurst M, Jang E. 2014. Improved traps for coconut rhinoceros beetle, \emph{Oryctes rhinoceros} Entomological Society of America, Portland, OR, 2014 Nov 19. Available from: \url{https://zenodo.org/record/165763}

% 2013

\bibitem{}Moore A, Marler T, Miller RH, Yudin LS 2013. Biological Control of Cycad Scale, Aulacaspis yasumatsui, Attacking Guam ? s Endemic Cycad , Cycas micronesica. In: 4th International Symposium on Biological Control [Internet]. Chile; 2013. Available from: \url{http://guaminsects.net/anr/sites/default/files/Moore et al. - 2013 - Biological Control of Cycad Scale, Aulacaspis yasumatsui, Attacking Guam ? s Endemic Cycad , Cycas micronesica.pdf}

\bibitem{}Moore A, Miller RH, Marler TE, Lee S. Yudin 2013. A coalition of invasive species attacks Guam?s native cycads. In: Entomoligical Society of America Annual Meeting [Internet]. Austin, Texas; 2013. Available from: \url{http://guaminsects.myspecies.info/sites/guaminsects.myspecies.info/files/cycas_poster_2013_0.pdf}

\bibitem{}Moore A, Miller RH, Marler TE 2013. Biological control of cycad scale, Aulacaspis yasumatsui, attacking Guam?s endemic cycad, Cycas micronesica. In: Entomological Society of America Annual Meeting [Internet]. Austin, Texas; 2013. Available from: \url{http://guaminsects.myspecies.info/sites/guaminsects.myspecies.info/files/CycadScaleBiocontrolAustin.pdf}

% 2012

\bibitem{key-2}Moore, A. 2012. Guam as a source of new insects for
Hawaii. Pacific Entomology Conference. Conference Paper (oral presentation)

\bibitem{key-3}Moore, A. 2012. CRB is the BTS of the 21st Century.
Brown Treesnake Technical Working Group Meeting. Conference Paper
(oral presentation)

\bibitem{key-4}Moore, A. 2012. Insect pests of ironwoods. Ironwood
Decline Conference, Guam. Conference Paper (oral presentation)

\bibitem{key-5}Moore, A. 2012. Insect pests of trees on Guam. Ironwood
Decline Conference, Guam. Conference Paper (oral presentation)

\bibitem{key-6}Moore, A. 2012. Update on the Guam coconut rhinoceros
beetle eradication project. Western Micronesia Invasive Species Committee
Annual Meeting. Conference Paper (oral presentation)

\bibitem{key-21}Moore, A. 2012. Update on the Guam coconut rhinoceros
beetle eradication project. Guam Invasive Species Council. Conference
Paper (oral presentation)

\bibitem{key-20}Moore, A, Quitugua R. 2011. Challenges of eradicating
coconut rhinoceros beetle, \emph{Oryctes rhinoceros}, on Guam. Society
of American Foresters Annual Conference. Conference Paper (oral presentation)

\bibitem{key-19}Moore, A. 2011. Update on the Guam coconut rhinoceros
beetle eradication project. Entomological Society of America Pacific
Branch Annual Meeting. Conference Paper (oral presentation)

\bibitem{key-18}Moore, A. 2011. Evaluation of a Scratchpad Template
as an Online Database for the University of Guam Insect Collection.
Entomological Collections Network Annual Conference. Conference Paper
(oral presentation)

\bibitem{key-17}Miller, RH, Moore A, Reddy GVP. 2011. The invasion
of Pacific islands: some thought on invasive species, insular ecosystems,
and human impact in the western Pacific. Entomological Society of
America Pacific Branch Annual Meeting. Conference Paper (oral presentation)

\bibitem{key-16}Moore, A. 2011. An update on the Guam coconut rhinoceros
beetle eradication project. Western Micronesia Invasive Species Committee
Annual Meeting. Conference Paper (oral presentation)

\bibitem{key-15}Moore, A. 2011. Containing the rhinoceros beetle
outbreak on Guam. International Plant Protection Congress. Conference
Paper (oral presentation)

\bibitem{key-7}Mersha, Z, Schlub RL, Spaine P, Smith J, Nelson S,
Moore A, McConnell J, Pinyopusarerk K, Nandwani D, Badilles A. 2010. Pre
and post January 2009 Guam ironwood, \emph{Casuarina equisetifolia},
tree decline conference. Conference Paper (oral presentation)

\bibitem{key-8}Moore, A. 2010. Update on the Guam Coconut Rhinoceros
Beetle Eradication Project. Entomological Society of America Annual
Meeting. Conference Paper (oral presentation)

\bibitem{key-9}Mersha, Z, Schlub RL, Moore A. 2009. The state of
ironwood, \emph{Casuarina equisitifolia} ssp. \emph{equisitifolia},
decline on the Pacific island of Guam. American Phytopathological
Society. Conference Paper (poster presentation)

\bibitem{key-10}Moore, A, Miller RH, Marler TE. 2009. Guam's native
cycads attacked by a coalition of invasive species. Entomological
Society of America Annual Meeting. Conference Paper (poster presentation)

\bibitem{key-11}Kirsch, P, Moore A, Kirsch C, Oluput G. 2009. Q-TRAP:
In-transit detection of bioinvasive insects in intermodal shipping
containers. 6th International Integrated Pest Management Symposium.
Conference Paper (poster presentation)

\bibitem{key-12}Kirsch, P, Wan E, Hunt J, Moore A. 2009. Monitoring
and automatic classification of flying insects. 6th International
Integrated Pest Management Symposium. Conference Paper (poster presentation)

\bibitem{key-13}Moore, A. 2008. Attempted eradication of the coconut
rhinoceros beetle, \emph{Oryctes rhinoceros}, (Scarabaeidae), a recently
arrived invasive species on Guam. Entomological Society of America
Annual Meeting. Conference Paper (oral presentation)

\bibitem{key-14}Mankin, RW, Moore A, Samson PR, Chandler KJ. 2008. Acoustic
characteristics of rhino beetle stridulations. Entomological Society
of America Annual Meeting. Conference Paper (oral presentation)

\bibitem{key-4}Moore, A., C. Apperson, J. McLaughlin, P. Kirsch \&
D. Czokajlo. Automated classification of morphologically identical
mosquito sibling species using wingbeat harmonics. Poster presentation
at the Annual Meeting of the Entomological Society of America, San
Diego, December, 2007.

\bibitem{key-38}Moore, A. \& R. H. Miller. Establishment of the Lady
Beetle, \emph{Rhyzobius lophanthae}, for biological control of the
Asian cycad scale, \emph{Aulacaspis yasumatsui} on Guam. Annual Meeting
of the Regional Biological Control Project. Kona, Hawaii, October
2007.

\bibitem{key-37}Moore A. Environmental Effects of Military Presence
on Guam. Keynote speaker; Annual Meeting of Land Grant Financial Officers,
Guam, August 2007.

\bibitem{key-36}Moore A. Invasive Insects on Guam. Guest speaker;
TSTAR Economics of Invasive Species Workshop. Guam, February, 2006.

\bibitem{key-35}Moore A. FAST-ID: Instrumentation for Automated Classification
of Flying Insects Using Optically-Sensed Wingbeat Waveforms. Western
Pacific Tropical Research Center Conference, August, 2006.

\bibitem{key-34}Moore A. FAST-ID: Instrumentation for Automated Classification
of Flying Insects Using Optically-Sensed Wingbeat Waveforms. Guest
speaker, Hawaiian Entomological Society, Honolulu, Hawaii, January,
2006.

\bibitem{key-33}Moore, A. Development of an optical flying insect
detection and identification system (OFIDIS).{[}poster{]} International
Conference on Integrated Pest Management, Toronto 2002.

\bibitem{key-32}Moore, A. Development of an optical flying insect
detection and identification system (OFIDIS). Entomological Society
of Canada Annual Meeting, Niagara Falls 2001.

\bibitem{key-31}Moore A. Development of an optical flying insect
detection and identification system (OFIDIS). Joint Annual Meeting
of the Entomological Society of America and the Entomological Society
of Canada, Montreal, 2000.

\bibitem{key-30}French, M., J. Miller \& A. Moore. Optical flying
insect detection and identification system (OFIDIS): Calibration and
detection of insects in the aquatic and forest-edge setting. Joint
Annual Meeting of the Entomological Society of America and the Entomological
Society of Canada, Montreal, 2000.

\bibitem{key-29}Moore, A. Development of an optical flying insect
detection and identification system (OFIDIS). Symposium 5.1: Technologies
for Movement and Migration Research ; XXI International Congress of
Entomology, Brazil, 2000. 

\bibitem{key-28}Moore, A. \& R. H. Miller. Automated identification
of optically sensed aphid wingbeat waveforms. Entomological Society
of America Annual Meeting, Atlanta, 1999.

\bibitem{key-27}Miller, R., K. Pike, P. Stary, A. Moore. Aphids and
aphidiid parasitoids in the Mariana Islands of Guam, Saipan, Tinian,
and Rota {[}poster{]}. Entomological Society of America Annual Meeting,
Atlanta, 1999.

\bibitem{key-26}Miller, R. H., K. S. Pike, P. Stary \& A. Moore.
Pacific island (Guam, Saipan, Tinian) aphids and associated parasitoids.
Entomological Society of America Pacific Branch Meeting, Eugene, Oregon,
1999. {[}poster{]} 

\bibitem{key-25}Moore, A. Automated monitoring of insect flight activity
in the field using a photosensor. Entomological Society of America
Annual Meeting, Las Vegas, 1998.

\bibitem{key-24}Miller, R. H., K. S. Pike, A. Moore \& P. Stary.
Opportunity for biological control of aphids in the Mariana Islands.
Entomological Society of America Annual Meeting, Las Vegas, 1998.
{[}poster{]}

\bibitem{key-23}Moore, A. Development of a data acquisition system
for long-term outdoor recording of insect flight activity using a
photosensor. 13th Conference on Biometeorology and Aerobiology, Albuquerque,
1998.

\bibitem{key-22}Moore, A. Automated monitoring of flying insects
using optically-sensed wingbeat waveforms. Entomological Society of
America Annual Meeting, Nashville, 1997. 

\bibitem{key-21}Moore, A. \& J. W. Brown. Automated monitoring of
free-flying insects using wingbeat waveforms. XX International Congress
of Entomology, Florence, 1996. {[}poster{]} 

\bibitem{key-20}Moore, A. Fruit flies in the Marianas: Past, Present,
\& Future. III Regional Conference on Agricultural Development in
Micronesia. Saipan, 1993. 

\bibitem{key-19}Moore, A. Population dynamics of \emph{Bactrocera}
fruit flies on Saipan. VII Pacific Science Inter-Congress, Okinawa,
1993. 

\bibitem{key-18}Chiu, C. H. \& A. Moore. Biological control of the
Philippine lady beetle, \emph{Epilachna philippinensis} Dieke (Coleoptera:
Coccinelidae), on solanaceous plants by introducing the parasitoid,
Pediobius foveolatus Crawford (Hymenoptera: Eulophidae), on Saipan.
XIX International Congress of Entomology, Beijing, 1992. 

\bibitem{key-17}Moore, A. Identification of flying insects using
an artificial neural network to recognize wingbeat spectra. XIX International
Congress of Entomology, Beijing, 1992. {[}poster{]} 

\bibitem{key-16}Gruenhagen, N. M., E. A. Backus, D. E. Ullman \&
A. Moore. A computerized system for acquiring and measuring waveforms
from AC electronic insect feeding monitors. XIX International Congress
of Entomology, Beijing, 1992. {[}poster{]} 

\bibitem{key-15}Moore, A. Automatic identification of flying insects
using an artificial neural network. Pacific Science Association, Guam,
1990. 

\bibitem{key-14}Moore, A. \& M. W. Johnson. A decision model for
watermelon IPM in Guam. Agricultural Development in the American Pacific
Crop Protection Conference, Guam, 1990. 

\bibitem{key-13}Cho, J.J., D. E. Ullman, T. L. German, D. Custer
\& A. Moore. Detection of cucurbit viral diseases in Hawaii. Agricultural
Development in the American Pacific Crop Protection Conference, Honolulu,
1989. 

\bibitem{key-12}Yudin, L. S., B. E. Tabashnik, W. C. Mitchell, \&
A. Moore. Predicting tomato spotted wilt incidence in lettuce. International
Conference on Tomato Spotted Wilt, Honolulu, 1989. 

\bibitem{key-11}Moore, A. \& B.E. Tabashnik. Monitoring insect landing
activity using a digital balance interfaced with a microcomputer.
Entomological Society of America, National Meeting, Boston, MA, 1987. 

\bibitem{key-10}Moore, A. Auto-amputation in diamondback moths: a
new form of insecticide resistance? Tester Symposium, Honolulu, HI,
1987. 

\bibitem{key-9}Moore, A. \& B.E. Tabashnik. Behavioral responses
of adult diamondback moths to pyrethroid residues. Entomological Society
of America, National Meeting, Reno, NA, 1986. 

\bibitem{key-8}Moore, A. Automated identification of flying insects
by analysis of wingbeat harmonics. Entomological Society of America,
Pacific Branch Meeting, Honolulu, HI, 1985. (Awarded second prize
in student paper competition) 

\bibitem{key-7}Moore, A. \& S. H. Gage. Fitting curves to phenology
data using an optimization technique. Entomological Society of America,
National Meeting, Detroit, MI., 1983. 

\bibitem{key-6}Moore, A. \& S. H. Gage. The Cooperative Crop Monitoring
System as a Potential Source of Data for Pest Phenology Models. Entomological
Society of America, North Central Branch Meeting, St. Louis, MO.,
1983. 
\end{thebibliography}

\end{document}
